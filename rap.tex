\documentclass{article}

\usepackage[french]{babel}
\usepackage[T1]{fontenc}
\usepackage{times}

\title{FIIL - Isabelle - Rendu de Projet}
\author{Jules Penuchot \& Antoine Lanco}

\begin{document}

\maketitle

\begin{abstract}

Les projets qui nous ont été donnés consistent à faire de la preuve en Isabelle. Nous étions censés être faire le deuxième sujet à la base (à savoir celui sur les Arbres Binaires de Décision), que nous avons entamé mais nous nous sommes finalement rabattus sur premier sujet, à savoir celui sur les Expréssions Régulières.

Nous avons décidé de nous rabattre sur le premier sujet en raison des difficultés pratiques que nous avons rencontrées sur les preuves du deuxième (BDT). Nous avons toutefois réussi à compléter les preuves du deuxième sujet (RegExp).

\end{abstract}

\section{Expressions Régulières}

\subsection{Introduction au sujet}



\section{Arbres de Décision}

\subsection{Introduction au sujet}

En informatique, un graphe de décision binaire ou diagramme de décision binaire
est une structure de données utilisée pour représenter des fonctions booléennes,
ou des questionnaires binaires. On utilise les BDD pour représenter des
ensembles ou des relations de manière compacte / compressée.

Nous avons commencé par nous approprier la strcutur du BDT et les fonctions qui permete de les manipuler. Puis nous avons implementer 


\subsection{Technique}



\end{document}
