\documentclass{article}

\usepackage[french]{babel}
\usepackage[T1]{fontenc}
\usepackage{times}

\title{FIIL - Isabelle - Rendu de Projet}
\author{Jules Penuchot \& Antoine Lanco}

\begin{document}

\maketitle

\begin{abstract}

Les projets qui nous ont été donnés consistent à faire de la preuve en Isabelle. Nous étions censés être faire le deuxième sujet à la base (à savoir celui sur les Arbres Binaires de Décision), que nous avons entamé mais nous nous sommes finalement rabattus sur premier sujet, à savoir celui sur les Expréssions Régulières.

Nous avons décidé de nous rabattre sur le premier sujet en raison des difficultés pratiques que nous avons rencontrées sur les preuves du deuxième (BDT). Nous avons toutefois réussi à compléter les preuves du deuxième sujet (RegExp).

\end{abstract}

\section{Expressions Régulières}

Ce sujet consiste à prouver un compilateur d'expressions régulières, puis de l'étendre en ajoutant l'opérateur "interleave" tout en prouvant que son ajout ne casse pas la régularité des expressions.

\subsection{Preuves sur les expressions régulières}

Les preuves ont été effectuées pour les lemmes suivants :

\begin{itemize}


\item steps\_epsclosure

  Succès de sledgehammer

\item in\_steps\_epsclosure

  Succès de sledgehammer

\item epsclosure\_steps

  Induction requise sur w, non trouvée par sledgehammer. La condition de base et la récurrence ont été prouvées avec succès grâce à sledgehammer

\item steps\_union

  Preuve par induction incomplète, le deuxième cas à prouver a été prouvé de manière similaire au premier ("by blast+").

\item sound\_and\_complete

  Cette preuve fut la plus compliquée, elle se fait par induction sur rexp en spécifiant un w arbitraire. Nous n'avons pas réussi à la terminer.

\item delta\_conv\_steps

  Prouvée par induction sur w

\item accepts\_conv\_steps

  Prouvée par simplification

\end{itemize}

\subsection{Opérateur interleave}


\section{Arbres de Décision}

\subsection{Introduction au sujet}



\end{document}
